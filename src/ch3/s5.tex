\wde{3.5.1 Relation} A relation $R$ on a set $X$ is a subset $R \subseteq X \times X$. $R$ is an equivalence relation on $X$ iff. the following hold:
(1) Reflexivity: $xRx$;
(2) Symmetry: $xRy \Leftrightarrow yRx$;
(3) Transitivity: $(xRy \land yRz) \rightarrow xRz$.
\wde{3.5.3 Equiv Class} Suppose that $\sim$ is an ER on $X$. For $x \in X$ the set $E(x) := \{z \in X : z \sim x\}$ is called the equivalence class of $x$. A subset $E \subseteq X$ is an EC for our ER if there is an $x \in X$ for which $E = E(x)$. An element of an EC is called a representative of the class. A subset $Z \subseteq X$ containing precisely one element from each EC is called a system of representatives for the ER.
\wde{3.5.5 Set of ECs} Given the ER $\sim$ on $X$ we denote the set of ECs, which is a subset of $\mathcal{P}(X)$, by $(X/\sim) := \{E(x) : x \in X\}$. There is a canonical surjection $\mathrm{can} : X \to (X/\sim); x \mapsto E(x)$.
\wde{3.5.5 Quotient Vector Space} Given a VS $V$ and a subspace $W \subseteq V$, define an ER $x \sim y$ iff $x - y \in W$ s.t. the EC of $x \in V$ is the subset $E(x) = W + x \subset V$. The quotient set $V/\sim = V/W$ is a VS with $\lambda E(x) = E(\lambda x)$ for $\lambda \in F$, and $\mathrm{can} : V \to V/W$ a surjective linear map with kernel $\mathrm{can}^{-1}(0) = W$. The VS $V/W$ is the quotient VS of $V$ by the subspace $W$. If $V$ is finite dimensional then so are $W$ and $V$, and $\dim(V/W) = \dim(V) - \dim(W)$.
\wde{1.7.1 Linear Map} A map $f : V \to W$ is called ($F$-)linear (or a homomorphism of $F$-vector spaces) if $\forall \vec{v}_{1}, \vec{v}_{2} \in V$ \& $\lambda \in F$ we have $f(\vec{v}_{1} + \vec{v}_{2}) = f(\vec{v}_{1}) + f(\vec{v}_{2})$ \& $f(\lambda \vec{v}_{1}) = \lambda \vec{v}_{1}$. If $f$ is bijective, we call the map an isomorphism of VSs. A homomorphism from a VS to itself is called an endomorphism. If an endomorphism is also an isomorphism, we call it an automorphism. The set of all homomorphisms is denoted $\Hom_{F}(V,W) \subseteq \Maps_{F}(V,W)$
\wde{1.7.5 Fixed Point} A point that is sent to itself by a map. Given $f : X \to X$, we denote the set of fixed points by $X^{f} = \{x \in X : f(x) = x\}$.
\wde{1.7.6 Complement} Subspaces $U,W$ are complementary if addition defines a bijection $U \times W \iso V$.
Note: iff $V = U + W$ and $U \cap W = \{0\}$.
\wde{1.7.6 (Direct) Sum} For subspaces $V_{1}, \dots, V_{n}$ of $V$, the subspace they generate is called the sum, denoted $V_{1} + \dots + V_{n}$. If the homomorphism given by $V_{1} + \dots + V_{n} \to V$ is injective, we call the sum direct (denoted $V_{1} \oplus \dots \oplus V_{n}$).
\wt{1.7.7 Classification of Vector Spaces by Dimension} A VS over $F$ is isomorphic to $F^{n}$ iff. it has dimension $n$.
\wl{1.7.8 Linear Maps \& Bases} Let $B \subset V$ be a basis. Restriction of a map gives a bijection $\Hom_{F}(V, W) \iso \Maps(B, W); f \mapsto f |_{B}$. i.e. each linear map determines \& is completely determined by the values it takes on a basis.
\wpr{1.7.9 Inverses}
(1) Every injective linear map $f : V \hookrightarrow W$ has a \emph{left inverse}, i.e. a linear map $g : W \to V$ such that $g \circ f = \id_{V}$.
(2) Every surjective linear map $f : V \twoheadrightarrow W$ has a \emph{right inverse}, i.e. a linear map $g : W \to V$ such that $f \circ g = \id_{W}$.

\wde{4.1.1 $n$-th Symmetric Group} The group of bijections from
$\{1, ..., n\}$ to itself is denoted $\mathfrak{S}_n$ (or $S_n)$.
It has $n!$ elements.
A transposition is a bijection that swaps exactly two elements.
\wde{4.1.2} Inversion of $\sigma \in \mathfrak{S}_n$ is a pair $(i, j)$
s.t. $1 \le i < j \le n$ and $\sigma(i) > \sigma(j)$ (crossings on a diagram).
Length of $\sigma$ is the number of inversions, denoted $\ell$.
$\ell = |\{(i, j) : i < j$ but $\sigma(i) > \sigma(j)\}|$.
Sign of $\sigma$ is $\sgn(\sigma) = (-1)^{\ell}$.
Even $\sigma$ have sign $+1$, odd $\sigma$ have sign $-1$.
\wl{4.1.5 Multiplicativity of Sign}
$\forall \sigma, \tau \in S_n$, $\sgn(\sigma \tau) = \sgn(\sigma) \sgn(\tau)$.
\wde{4.1.6 Alternating group}
The set of even permutations is a subgroup of $S_n$ (denoted $A_n$).
It's the kernel of the group hom $\sgn : S_n \to \{+1, -1\}$.

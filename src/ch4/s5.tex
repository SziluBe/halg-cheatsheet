\wde{4.5.1 Eigenvalues, Eigenvectors, Eigenspaces}
Let $V$ an $F$-v s, $f: V \to V$ an endom. $\lambda \in F$ is an eigenvalue of $f$ iff
$\exists v \in V$ s.t. $f(v) = \lambda v$. Then $v$ is an eigenvec of $f$ w/ eigenval $\lambda$.
The eigenspace of $f$ w/ eigenval $\lambda$ is $E(\lambda, f) = \{v \in V : f(v) = \lambda v\}$.
This is a subsp of $V$.
\wt{4.5.4 Existence of Eigenvalues}
Each endom of a non-zero fin dim vec space over an algebraically closed field has an eigenvalue.
\wde{4.5.6 Characteristic Polynomial}
Let $R$ a comm ring, $A \in \text{Mat}(n; R)$.
Then the char poly of $A$ is the polynomial $\chi_A(x) = \text{det}(xI_n - A)$.
\wt{4.5.8 Eigenvals and Char Polys}
$F$ field, $A \in \text{Mat}(n; F)$.
The eigenvals of the lin map $A : F^n \to F^n$ are the roots of $\chi_A(x)$.
Note: $\chi_A(x) = x^n - \text{Tr}(A)x^{n-1} + ... + (-1)^n \text{det}(A)$.
\wde{Conjugate matrices}
$A, B \in \text{Mat}(n; R)$ are conjugate iff there exists $P \in \text{GL}(n; R)$ s.t. $B = P^{-1}AP$.
This is an eq reln on $\text{Mat}(n; R)$.
Note: char polys of conjugates are the same.
Note: endom of odd-dim real v s always has a real eigenval,
and if its det is positive and real then it has a positive real eigenval.

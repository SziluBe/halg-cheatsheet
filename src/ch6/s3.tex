\wl{6.3.1} There exist polynomials $Q_j(x) \in F[x]$ s.t. $\sum_{j=1}^sP_j(x)Q_j(x) = 1$.
\wde{6.3.2 Generalised Eigenspace} The general eigenspace of $\phi$ with eigenvalue $\lambda_i$, $E^{\mathrm{gen}}(\lambda_i, \phi)$ is the following subspace of $V$: $E^{\mathrm{gen}}(\lambda_i, \phi) = \left\{ \vec{v} \in V | (\phi - \lambda_i \id_V)^{a_i}(\vec{v}) = \vec{0} \right\}$. Each $a_i$ is a positive distinct integer. The actual eigenspace is the same, but all $a_i=1$. The dimension of $E^\mathrm{gen} (\lambda_i, \phi)$ is called the algebraic multiplicity of $\phi$ with eigenvalue $\lambda$, and the dimension of $E(\lambda_i, \phi)$ is the geometric multiplicity. 
\wde{6.3.4 Stability} Let $f : X \to X$ be a map from a set $X$ to itself. A subset $Y \subseteq X$ is stable under $f$ precisely when $f(Y) \subseteq Y$, that is if $y \in Y$ then $f(y) \in Y$.
\wpr{6.3.5 Direct Sum Decomposition} For each $1 \le i \le s$, let $\mathcal{B}_i = \left\{ \vec{v}_{ij} \in V | 1 \le j \le a_i \right\}$ be a basis of $E^{\mathrm{gen}}(\lambda_i, \phi)$, where $a_i$ is the algebraic multiplicity of $\phi$ wih eigenvalue $\lambda_i$, s.t. $\sum_{i=1}^s a_i = n$ is the dimension of $V$.
(1) Each $E^\mathrm{gen} (\lambda_i, \phi)$ is stable under $\phi$.
(2) For each $\vec{v} \in V$ there exists unique $\vec{v} \in E^\mathrm{gen} (\lambda_i, \phi)$ such that $\vec{v} = \sum_{i=1}^s \vec{v}_i$. In other words, there is a direct sum decomposition $V = \oplus_{i=1}^s E^\mathrm{gen} (\lambda_i, \phi)$ with $\phi$ restricting to endomorphisms of the summands $\phi_i = \phi| : E^\mathrm{gen} (\lambda_i, \phi) \to E^\mathrm{gen} (\lambda_i, \phi)$.
(3) $\mathcal{B} = \mathcal{B}_1 \cup \mathcal{B}_2 \cup \dots \cup B_s = \left\{ \vec{v}_{ij} | 1 \le i \le s, 1 \le j \le a_i \right\}$ is a basis of $V$. The matrix of the endomorphism $\phi$ with respect to this basis is given by the block diagonal matrix $\,_{\mathcal{B}}[\phi]_{\mathcal{B}} = \diag(B_1, B_2, \dots, B_s) \in \Mat(n; F)$ with $B_i = \,_{\mathcal{B}_i}[\phi_i]_{\mathcal{B}_i} \in \Mat(a_i, F)$.
\wl{6.3.6} For each $i$, define a linear map $\psi_i : \frac{W_i}{W_{i-1}} \to \frac{W_{i-1}}{W_{i-2}}$ by $\psi_i(\vec{w} + W_{i-1}) = \psi(\vec{w}) + W_{i-2}$ for $\vec{w} \in W_i$. Then $\psi_i$ is well-defined and injective.
\wl{6.3.7} Let $f : X \to Y$ be an injective linear map between $F$-VSs $X$ and $Y$. If $\left\{ \vec{x}_1, \dots, \vec{x}_t \right\}$ is a LI set in $X$, then $\left\{ f(\vec{x}_1), \dots, f(\vec{x}_t) \right\}$ is a LI set in $Y$.
\wl{6.3.8} The set of elements $\left\{ \vec{v}_{j,k} : 1 \le j \le m, 1 \le k \le d_k \right\}$ constructed in the algorithm above is a basis for $W$.
\wpr{6.3.9} Let $\mathcal{B}$ be the ordered bsis of $W$ constructed above ($\vec{v}_{jk} : 1 \le j \le m, 1 \le k \le d_j$). Then
$\,_{\mathcal{B}}[\psi]_{\mathcal{B}} = diag(\underbrace{J(m), \dots, J(m)}_{\text{$d_m$ times}}, \underbrace{J(m-1), \dots, J(m-1)}_{\text{$d_{m-1}-d_m$ times}}), \dots, \underbrace{J(1), \dots, J(1)}_{\text{$d_1 - d_2$} times}$.